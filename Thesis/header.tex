% !TEX root = thesis.tex
% ===============================================
% ===============================================
% 	LaTeX Header File - Defines all used packages and options
% ===============================================
% ===============================================
%
%
% ===============================================
% A. DOCUMENT CLASS (options and packets to use)
% ===============================================
\RequirePackage[ngerman=ngerman-x-latest]{hyphsubst}	% better german hyphenation rules
\documentclass[
	paper		= a4,	% Use A4 paper
	english,
%	ngerman,
	BCOR=4mm,		% Additional border for binding the document
	% BCOR depending on the number of pages
	% Bei 80g/m^2 Papier
	% 160 Seiten --> 8mm
	% 140 Seiten --> 7mm
	% 120 Seiten --> 6mm
	% ..
	% 60 Seiten --> 3mm
	% 40 Seiten --> 2mm
	% Bei 100g/m^2 Papier
	% 150 Seiten --> 9mm
	% 133 Seiten --> 8mm
	% 117 Seiten --> 7mm
	% 100 Seiten --> 6mm
	% 83 Seiten --> 5mm
	% 67 Seiten --> 4mm
	% 50 Seiten --> 3mm
	DIV=13,								% Proportion for empty and non-empty parts of a page
	%twoside,
	titlepage,							% Support for title environment
	headinclude		= true,				% Include header in typearea calculation
	footinclude		= false,			% Include footer in typearea calculation
	twoside			= false,			% Two-sided printing - true or false
	%cleardoublepage         = empty,	% Empty backsides for two-sided printing
	numbers 		= noenddot,			% Numbers (in headlines) without ending points
	parskip			= half,				% A half line to separate paragraphs
	chapterprefix	= false,			% Chapters with Prefix - true or false?
	appendixprefix	= false,			% Appendix with Prefix - true or false?
	headsepline,						% Line after header
	%headings 		= normal,			% Size of Headers (normal, small, big)
	bibliography	= totoc,			% Include Bibliography in Table of Contents
	listof 			= totoc,			% Include Lists in Table of Contents
	pagesize		= pdftex			% Define Typearea for pdftex usage
]{tudscrreprt} 			% load Koma-Script-Class for the CD of the TUD
\usepackage{scrhack}	% includes koma script patches for other packages
% ===============================================
%	Character coding and fonts for the document
% ===============================================
\usepackage[default,osfigures,scale=0.95]{opensans}	% font for the CD
\usepackage[utf8]{inputenc}							% UTF-8 Support
\usepackage[T1]{fontenc}							% T1 Fonts for font encoding
%\usepackage[ngerman]{babel}							% German language with new orthography
\usepackage[english]{babel}						% English language	
\usepackage[babel]{microtype}						% better layout of the text
% ===============================================
%	Package for quotation marks
% ===============================================
%\usepackage[babel=false, style=swiss]{csquotes}		% Swiss style quotes
%\usepackage[babel=false, german=quotes]{csquotes}		% German style quotes
\usepackage[babel=false, english=american]{csquotes}	% American style quotes
%\usepackage[babel=false, german=quotes]{csquotes}		% German style quotes
% ===============================================
%	Packages for additional table features
% ===============================================
\usepackage{array}			% Additional commands for tables
\usepackage{tabularx}		% Automatic row wide with special parameter
\usepackage{booktabs}		% Packet for scientific table environments
\usepackage{longtable}		% Support for tables over multiple pages
\usepackage{multirow}		% Support for spanning cells over multiple rows
\usepackage{longtable}		% For tables longer than one site
\usepackage{ltxtable}		% use X column from tabularx in longtable
\usepackage{tabu}		% For the symbol table only
% ===============================================
%	Packages for additional mathemtical features
% ===============================================
\usepackage{amsmath, amsthm, amssymb}	% ASM Mathematic packages
\usepackage{pifont}						% used for circled numbers
\usepackage{siunitx}					% Numbers with nice untis
% ===============================================
%	Packages for figures and captions
% ===============================================
\usepackage{pdfpages}	% Einbinden von PDF als komplette Seite
\usepackage{graphicx}	% Einbinden von Bildern				
\usepackage{caption}	% Packet to support captions for subfigures
\usepackage{subcaption}	% Packet to support subfigure environments
%\renewcommand\thesubfigure{(\alph{subfigure})}% für Klammern bei Referenz der Subfigure(z. B. Abbildung 3.1(a))
% ===============================================
%	Additional packages
% ===============================================
\usepackage{isodate}			% for several date formats
\usepackage{xcolor}				% Packet for colour text support
\usepackage{enumitem}			% Alternitive Item environment for customizing lists
\usepackage{flafter}			% Never insert figures before the actual reference
\usepackage[section]{placeins}	% Insert floatbarriers before every section
\usepackage{ellipsis}			% Corrects wrong spaces after dots
\usepackage{blindtext}
%\usepackage{easy-todo}			% for colored Todo-Notes in the PDF
% ===============================================
%	Packages for references and citation
% ===============================================
\usepackage[%
	backend=biber,		% for sorting the entries and citing information with UTF-8 support
	style=numeric,	% style for citation
	url=false,			% url-field of the bib is not printet in the references
	maxalphanames=1,	% 
	maxcitenames=1,		%
	]%
{biblatex}	%
\addbibresource{bib/references.bib}	% Adding the bib-file to reference, if need more copy this command
% ===============================================
%	Packages for special PDF features and References
% ===============================================
\usepackage[%
	pdfauthor={Julian Haase},			%
	pdftitle={Titel der Arbeit},		%
	pdfsubject={XXX Thesis},			%
	pdfkeywords={},						%
	pdfcreator={PdfLaTeX},				%
	pdfproducer	={LaTeX with hyperref},	%
	breaklinks=true,	%
	colorlinks=true,	%
	% 	PDF-options for PRINT OUT of the document			    
	%	urlcolor=black,     %
	%	citecolor=black,    %
	%	linkcolor=black,    %
	%	filecolor=black,	%
	% 	PDF-options for PDF file creation (e.g. coloured links)
	urlcolor=blue,      %
	citecolor=red,      %
	linkcolor=blue,     %
	filecolor=blue,		%		
	bookmarks,				%
	bookmarksopen,			%
	bookmarksopenlevel=1,	%
	bookmarksnumbered,		%	
	pdfpagelabels,			%
	pdfdisplaydoctitle, 	%
	pdfpagemode=UseOutlines,%
	linktoc=all, 			%
	]%
{hyperref}%
\usepackage[%
	open,			%
	openlevel=2, 	%
	]%
{bookmark}%
\usepackage[nameinlink,noabbrev]{cleveref} % simpler refs with \cref
% ===============================================
%	Packages for glossary (acronyms, symbols, etc.)
% ===============================================
\usepackage[%
	acronym,%
	symbols,%
	nomain,%
	nogroupskip,%
	toc,%
	section=chapter,%
	nostyles,%
	translate=babel,%
	xindy={language=german-din},%
	%automake,
]{glossaries}
\makeglossaries
%
%
% ===========================================================================
% B. SETTINGS (for the document and the packages)
% ===========================================================================
\pagestyle{headings}	% Headers with page title
\frenchspacing			% Limit the spacing after dots
\sisetup{locale = DE}	% german untis
%
%
% ===========================================================================
% C. NEW MAKROS AND ENVIRONMENTS (for later use in the document)
% ===========================================================================
%
% ===============================================
%	New Columtyps
% ===============================================
\newcolumntype{Y}{>{\centering\arraybackslash}X} % centered X columns for tablurarx
% ===============================================
%	\includegraphicstotab[..]{..}
% ===============================================
\newlength{\myx} % Saving the width of the figure
\newlength{\myy} % Saving the height of the figure
\newcommand\includegraphicstotab[2][\relax]{%
	% save the figure values
	\settowidth{\myx}{\includegraphics[{#1}]{#2}}%
	\settoheight{\myy}{\includegraphics[{#1}]{#2}}%
	% insert into to the cell of the table
	\parbox[c][1.1\myy][c]{\myx}{%
		\includegraphics[{#1}]{#2}}%
}%
% ===============================================
%	New Citation-Style [FirstAuthor Year]
% ===============================================
\renewcommand*{\labelalphaothers}{} %
\DeclareLabelalphaTemplate{%
	\labelelement{%
		\field[final]{shorthand}%
		\field{labelname}%
		\field{label}%
	}
	\labelelement{%
		\literal{\addhighpenspace}%
	}
	\labelelement{%
		\field{year}%
	}
}
% ===============================================
%	\newsymbol{label}{description}{symbol}{unit}
% ===============================================
\renewcommand*{\newsymbol}[5][]{%
	\newglossaryentry{#2}{%
		type=symbols,%
		name={#3},%
		description={\nopostdesc},%
		symbol={\ensuremath{#4}},%
		user1={\ensuremath{\mathrm{#5}}},%
		sort={#2},%
		#1%
	}%
}
\defglsentryfmt[symbols]{%
	\ifmmode%
	\glssymbol{\glslabel}%
	\else%
	\glsgenentryfmt~\glsentrysymbol{\glslabel}%
	\fi%
}
% ===============================================
%	Symbol table style for the glossary
% ===============================================
\newglossarystyle{symblongtabu}{%
	\renewenvironment{theglossary}{%
		\begin{longtabu}spread 0pt[l]{ccX<{\strut}l}%
		}{%
		\end{longtabu}%title
	}%
	\renewcommand*{\glsgroupheading}[1]{}%
	\renewcommand*{\glsgroupskip}{}%
	\renewcommand*{\glossaryheader}{%
		\toprule
		\bfseries Formelzeichen & \bfseries Einheit &
		\bfseries Bezeichnung & \bfseries Seite(n)
		\tabularnewline\midrule\endhead%
		\bottomrule\endfoot%
	}%
	\renewcommand*{\glossentry}[2]{%
		\glsentryitem{##1}% Entry number if required
		\glstarget{##1}{\glossentrysymbol{##1}} &
		\glsentryuseri{##1} &
		\glossentryname{##1} &
		##2\tabularnewline%
	}%
}
% ===============================================
%	Acronym table style for the glossary
% ===============================================
\newglossarystyle{acrotabularx}{%
	\renewenvironment{theglossary}{%
		\tabularx{\linewidth}{@{}lXl@{}}%
	}{%
		\endtabularx\par\bigskip%
	}%
	\renewcommand*{\glossaryheader}{}%
	\renewcommand*{\glsgroupheading}[1]{}%
	\renewcommand*{\glsgroupskip}{}%
	\renewcommand*{\glossentry}[2]{%
		\glsentryitem{##1}% Entry number if required
		\glstarget{##1}{\sffamily\bfseries\glossentryname{##1}} &
		\glsentrydesc{##1} &
		##2\tabularnewline
	}
}
% ===============================================
%	Macros for fixme entries etc.
% ===============================================
% Macro for fixme entries
\newcommand{\fixme}[1]{\textcolor{blue}{FIXME: #1}}
% Macro for change entries
\newcommand{\change}[1]{\textcolor{green}{CHANGE: #1}} 
% Macro for remark entries
\newcommand{\remarks}[1]{\textcolor{magenta}{REMARKS: #1}} 
\newcommand{\re}{\textcolor{cyan}{\textbf{[ref]}}} 
% A Makro to support rephrase suggestions
% use as follows: rephrase{original}{new(suggestion)}{reason}, for no reason leave last argument empty!
\definecolor{darkgreen}{rgb}{0.0,0.6,0.0}
\newcommand{\rephrase}[3]{\textcolor{red}{#1}\footnote{REPHRASE: #3} \textcolor{darkgreen}{#2}}
% ===============================================
%	Placeholder for figures
% ===============================================
\newcommand{\missingfigure}[2][]{%
	\par
	\noindent
	\begin{tikzpicture}
	\draw[fill=black!40, draw = white, line width=0pt](-2, -2.5) rectangle +(\columnwidth, 4cm);
	\draw (2, -0.3) node[right, text width=\columnwidth-4.5cm] {#2};
	\draw[red, fill=white, rounded corners = 5pt, line width=10pt](30:2cm) -- (150:2cm) -- (270:2cm) -- cycle;
	\draw (0, 0.3) node {Missing};
	\draw (0, -0.3) node {Figure};
	\end{tikzpicture}
}% Ending \missingfigure command
%
%
% ===========================================================================
% D. ACTIONS AND OPERATIONS
% ===========================================================================
%
% ===========================================================================
% EOF
% ===========================================================================
