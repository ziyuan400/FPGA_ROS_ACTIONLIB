% ===============================================
	% LaTeX Task Template
	%
	% Author: Julian Haase
	% E-Mail: Julian.Haase@tu-dresden.de
	% Date: 25.04.2018
	% Please contact the author in case of questions.
	% ===============================================
	\RequirePackage[ngerman=ngerman-x-latest]{hyphsubst}	% better german hyphenation rules
	\documentclass[ngerman]{tudscrreprt}	% load Koma-Script-Class for the CD of the TUD
	\usepackage{scrhack}	% includes koma script patches for other packages
	% ===============================================
	%	Character coding and fonts for the document
	% ===============================================
	\usepackage[default,osfigures,scale=0.95]{opensans}	% font for the CD
	\usepackage[utf8]{inputenc}	% UTF-8 Support
	\usepackage[T1]{fontenc}	% T1 Fonts for font encoding
	%\usepackage[ngerman]{babel}	% German language with new orthography
	\usepackage[english]{babel}	% English language	
	\usepackage{microtype}	% better layout of the text
	% ===============================================
	%	Watermark - only for the draft
	% ===============================================
	%\usepackage[firstpage]{draftwatermark}
	%\SetWatermarkText{\textbf{draft}}
	%\SetWatermarkScale{1.1}
	% ===============================================
	%	Additional packages
	% ===============================================
	\usepackage{isodate}	% for several date formats
	\usepackage{enumitem}\setlist{noitemsep}	% less space between the items
	\usepackage{tudscrsupervisor}	% package for supervising templates for the CD of the TUD
	\usepackage{hyperref}	% get some links in the final document
	
	\usepackage{setspace}
	
	\begin{document}	
	% ===============================================
	%	Heading for the chair
	% ===============================================
	%\faculty{Fakultät Informatik}
	\faculty{Faculty of Computer Science}
	%\institute{Institut für Technische Informatik}
	\institute{Institute of Computer Engineering}
	%\chair{Professur für Adaptive Dynamische Systeme}
	\chair{Chair of Adaptive Dynamic Systems}
	%\headlogo{../../ADS/ADS_Logo.png}
	% ===============================================
	% ===============================================
	% ===============================================
	%	Change this for the given case
	% ===============================================
	% ===============================================
	% ===============================================
	\title{%
    Hardware Implementation of Preemtable Scheduling Approaches for the Robot Operating System (ROS) via actionlib
	}
	\thesis{master} % diploma, bachelor, student -> for details take a look at table 2.1 in tudscr.pdf
	\author{
		Ziyuan Zhang
		\matriculationnumber{4659711}%
		\course{Master Computer Science}%
		\discipline{Technische Informatik}%
	}
	\matriculationyear{2016}
	\issuedate{04.01.2022}
	\duedate{07.06.2022}
	\supervisor{Ariel Podlubne}
	%\referee{Prof. Dr. Diana Göhringer}
	% \professor{Prof. Dr. Diana Göhringer and Prof. Dr. rer. nat. habil. Uwe Aßmann}
	\professor{Prof. Dr. Diana Göhringer}
	\taskform[pagestyle=empty]{%
		% Description of the thesis -> main objective
		The aim of this research work focuses on the actionlib stack from ROS/ROS2. 
        The main question to answer is whether it is possible to have a reliable scheme for preemtable tasks following the actionlib for Field Programmable Gate Arrays (FPGAs).
        This would require an analysis of the state-of-the-art on Hardware (HW) schedulers, with a focus on preemtable tasks.
        The impact of the HW implementation of a client and server on HW must be compared to the standard one on Software (SW) and be quantified to draw conclusions.
        It is important to evaluate how the exchange of data between HW and SW should be and whether it is benefitial to have the Finite State Machines (FSMs) completley on HW.
        The benefits must be highlighted. 
        Besides, the impact on the preemtion of HW accelerators must be considered.
        Finally, the outcome should be a \textbf{generalizable} approach, considering that a system will have \textbf{multiple} HW-actions (accelerators).
	
	%	{\fontsize{60}{48} \selectfont \textcolor{red}{Replace with}}
	%	
	%	\noindent{\fontsize{60}{48} \selectfont \textcolor{red}{scanned original!}}
	}{% the main points of the thesis
	\vspace{-0.1cm}\item Research state-of-the-art for similar techniques
        \begin{itemize}
            \vspace{-0.1cm}\item Real-time and preemtable HW schedulers
            \vspace{-0.1cm}\item VHDL implementations of FSMs 
        \end{itemize}
		\vspace{-0.1cm}\item Become familiar with the Robot Operating System (ROS)
        \begin{itemize}
            \vspace{-0.1cm}\item Understand the differences among \textbf{topics}, \textbf{services} and \textbf{actions}
            \vspace{-0.1cm}\item Propose usecases for actionlib as proof of concept for one and multiple actions
        \end{itemize}
		\vspace{-0.1cm}\item Implement Client and Server FSMs on HW (VHDL)
        \begin{itemize}
            \vspace{-0.1cm}\item \textbf{Proof of concept} for \textit{one action}
            \vspace{-0.1cm}\item \textbf{Generalize} for \textit{multiple actions} at the same time
            \begin{itemize}
                \vspace{-0.1cm}\item Are multiple "actionlib" FSMs needed? If so, define the interaction among them
                \vspace{-0.1cm}\item Is a \textit{general} FSM needed to coordinate all "child" FSMs?
            \end{itemize}
        \end{itemize}
		\vspace{-0.1cm}\item Define Performance Metric (e.g., latency, throughput, tasks' and scheduler-FSMs- delays, HW resources) as in the state-of-the-art.
		\vspace{-0.1cm}\item Evaluate the implementation and compare it using the performance metric with traditional SW actionlib.
	}
	\end{document}
Event Timeline
