% !TEX root = thesis.tex
% ===============================================
% ===============================================
% 	LaTeX Chapter File - Main Text
% ===============================================
% ===============================================
\chapter{Latex syntax examples}

\textcolor{red}{This chapter must be removed in the final version to be printed.}

\section{Test acronyms}

first time \gls{fpga} and \gls{noc}. Second time \gls{fpga} and \gls{noc}.

%\section{Test symbols}

%Test in text: \gls{f}, \gls{U} and \gls{I}
%
%Test in equation:
%\begin{align}
%	\gls{U} &= \SI{50}{\volt} \\
%	\gls{I} &= \SI{10}{\milli\ampere} \\
%	\gls{f} &= \SI{10}{\mega\hertz}
%\end{align}

\section{Test figures}

Here are some examples on how to place figures. \Cref{fig:tiefpass} is just a single figure.

\begin{figure}[htb]
	\centering
	\includegraphics[width=.5\linewidth]{figures/tiefpass.png}
	\caption{Idealer Tiefpass}
	\label{fig:tiefpass}
\end{figure}

But \cref{fig:functions} is one figure-environment with two figures. You can reference to each of the figures in the environment. \Cref{fig:si-function} is a Si-function and \cref{fig:dreieck} is a triangular function.

\begin{figure}[htb]
	\centering
	\begin{subfigure}[b]{0.4\textwidth}
		\includegraphics[width=\textwidth]{figures/Si-function.png}
		\caption{Si-Funktion}
		\label{fig:si-function}
	\end{subfigure}
	~ %add desired spacing between images, e. g. ~, \quad, \qquad, \hfill etc. 
	%(or a blank line to force the subfigure onto a new line)
	\begin{subfigure}[b]{0.4\textwidth}
		\includegraphics[width=\textwidth]{figures/dreieck.png}
		\caption{Dreieck}
		\label{fig:dreieck}
	\end{subfigure}
	\caption{Pictures of functions}
	\label{fig:functions}
\end{figure}

Figures are always placed \emph{after} referencing them.

\section{Test tables}

This is an example of a simple table. \Cref{tab:flocklab_clock_offset} shows random stuff.
%
\begin{table}[htb]
	\centering
	\caption{A simple table.}
	\begin{tabular}{l c c}
		\toprule
		Node id & DCO frequency & DCO offset\\ \midrule
		11, 17, 23 & 4,131,389 Hz & -1.5\% \\
		4, 10, 16, 22, 28 & 4,152,361 Hz & -1\% \\
		1, 3, 7, 13, 15, 19, 25, 27, 33 & 4,194,304 Hz & 0\% \\
		6, 18, 24 & 4,236,247 Hz & +1\% \\
		2, 8, 14, 20, 26, 32 & 4,257,218 Hz & +1.5\% \\
		\bottomrule
	\end{tabular}
	\label{tab:flocklab_clock_offset}
\end{table}

As figures, tables should be places \emph{after} referencing them.

\section{Test citation}

\cite{Goehringer2013}, \cite{Logvinenko2014}, \cite{Ogras2013}, \cite{Rantala2006}
