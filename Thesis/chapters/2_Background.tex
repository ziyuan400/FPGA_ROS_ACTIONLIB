\chapter{Background}
\label{sec:background}

Hennessy, John L. and Patterson, David A. predicted in their report A New Golden Age for Computer Architecture{\cite{Hennessy2019}} that the new era of computer architecture will expect the same rapid improvement as RISC, the last golden age and this time in terms of cost, security, and energy, as well as in performance. The most important method is high-level, \gls{dsl} and architectures. The development of \gls{dsl} freeing architects from the chains of commercial proprietary instruction sets, and require an agile chip development. These changes will lead to a new golden age for computer architects.

AI is perhaps the most important current application space, not only for \gls{fpga}, but all segments of the semiconductor industry. \gls{dsa} are our hope to feed the huge need on AI compute at the end of Dennard scaling and Moore’\gls{dsa}s Law. \gls{dsa}s exploit for the specific domain a more efficient form of parallelism. \gls{dsa}s can improve performance by making more effective use of the memory hierarchy. Low precision like int8 or int4 are proved adequate for many AI computations and \gls{dsa}s can use less precision to gain efficiency and save energy when it is adequate. \gls{dsa}s benefit from targeting software programs written and optimized in \gls{dsl}s that expose more parallelism.

Agile hardware development start with a software simulator, the easiest and quickest place to find bugs and make changes if a simulator could satisfy an iteration of your project. The next level is  \gls{fpga} hardware that can run hundreds of times faster than a detailed software simulator. Amazon Web Services offers  \gls{fpga}s as a service in the cloud, so architects can try  \gls{fpga}s without even to buy a real hardware and set up a lab.

\gls{rtos} is an efficient tool to manage the software and make it easy to distribute tasks among developers\cite{Stallings2008}. Even for small-scale embedded systems, using an software  \gls{rtos} lets project leaders efficiently partition the entire software design into smaller modular tasks that individual developers can handle. Hardware \gls{rtos} enables other developers to develop the hardware drivers and components by providing the abstraction layer and therefore also provides better and safer synchronization.

Robots can be utilized in assembly lines where they interact with human workers in various ways\cite{Cashmore2015}. By completing certain tasks faster and with higher precision, they can have a positive impact on both production cost and output quality. The increasingly complex interaction between human and robot workers intensifies the demand for effective and flexible programming tools. In the past, programming environments and languages for robots were often targeted towards engineers and software developers. As such, they typically favored fine-grained control over all low-level operations performed by the robot over accessibility or intuitive usability. The resulting complexity of the tools causes them to often require years of education and training to use them effectively. Especially in smaller companies, this can slow down the adoption of robotics significantly.



ROS is an open source robot operating system\cite{quigley2009ros}. ROS is not an operating system in the traditional sense that designed for processing management and scheduling; rather, it provides a concept of a structured communications layer above all the host heterogeneous compute cluster with its own operating systems independently. A system built using ROS may consists of a number of processes, potentially in a distribute system on a number of different hosts, connected at run-time dynamically in a peer-to-peer topology.
